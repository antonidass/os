\chapter{Функции обработчика прерывания от системного таймера в защищённом режиме}

\section{Функции обработчика прерывания от системного таймера в защищённом режиме для ОС семейства UNIX/Linux}

Обработчик прерывания от системного таймера \textbf{по тику} выполняет следующие задачи:
\begin{itemize}
    \item инкрементирует счетчик тиков аппаратного таймера;
    \item декременирует квант текущего потока;
    \item обновляет статистику использования процессора текущим процессом;
    \item обновляет часы и другие таймеры системы;
    \item декрементирует счетчик времени до отправления на выполнение отложенного вызова (если счетчик достиг нуля, то выставление флага для обработчика отложенного вызова).
\end{itemize}

Обработчик прерывания от системного таймера \textbf{по главному тику} выполняет следующие задачи:
\begin{itemize}
    \item регистрирует отложенные вызовы функций, относящиеся к работе планировщика, такие как пересчет приоритетов;
    \item пробуждает в нужные моменты системные процессы, такие как \texttt{swapper} и \texttt{pagedaemon}.
        Под пробуждением понимается регистрация отложенного вызова процедуры \texttt{wakeup}, которая перемещает дескрипторы процессов из списка ``спящих'' в очередь готовых к выполнению.
    \item декрементирует счётчик времени, оставшегося времени до посылки одного из следующих сигналов:
        \begin{itemize}
            \item \texttt{SIGALRM} – сигнал, посылаемый процессу по истечении времени, предварительно заданного функцией \texttt{alarm()};
            \item \texttt{SIGPROF} –  сигнал, посылаемый процессу по истечении времени заданного в таймере профилирования;
            \item \texttt{SIGVTALRM} –  сигнал, посылаемый процессу по истечении времени, заданного в ``виртуальном'' таймере.
        \end{itemize}
\end{itemize}

Обработчик прерывания от системного таймера \textbf{по кванту} выполняет следующие задачи:
\begin{itemize}
    \item посылает текущему процессу сигнал \texttt{SIGXCPU}, если тот превысил выделенную ему квоту использования процессора.
\end{itemize}


\section{Функции обработчика прерывания от системного таймера в защищённом режиме для ОС семейства Windows}

Обработчик прерывания от системного таймера \textbf{по тику} выполняет следующие задачи:
\begin{itemize}
    \item инкрементирует счётчика системного времени;
    \item декрементирует кванта текущего потока на величину, равную количеству тактов процессора, произошедших за тик (если количество затраченных потоком тактов процессора достигает квантовой цели, запускается обработка истечения кванта);
    \item декрементирует счетчиков времени отложенных задач;
    \item если активен механизм профилирования ядра, инициализирует отложенный вызов обработчика ловушки профилирования ядра путем постановки объекта в очередь DPC: обработчик ловушки профилирования регистрирует адрес команды, выполнявшейся на момент прерывания.
\end{itemize}

Обработчик прерывания от системного таймера \textbf{по главному тику} выполняет следующие задачи:
\begin{itemize}
    \item освобождает объект ``событие'', который ожидает диспетчер настройки баланса.
\end{itemize}

Обработчик прерывания от системного таймера \textbf{по кванту} выполняет следующие задачи:
\begin{itemize}
    \item инициализирует диспетчеризацию потоков путем постановки соответствующего объекта в очередь \texttt{DPC}.
\end{itemize}


\chapter{Пересчёт динамических приоритетов}

Как в ОС семейства \texttt{UNIX/Linux} так и в ОС семейства \texttt{Windows} могут \textbf{динамически пересчитываться} только \textbf{приоритеты пользовательских процессов}.

\section{Пересчёт динамических приоритетов в операционных системах UNIX/Linux}

В современных системах \texttt{UNIX/Linux} ядро является вытесняющим – процесс в режиме ядра может быть вытеснен более приоритетным процессом в режиме ядра. Ядро было сделано вытесняющим для того, чтобы система могла обслуживать процессы реального времени, такие как аудио и видео.

Очередь готовых к выполнению процессов формируется согласно приоритетам процессов и принципу вытесняющего циклического планирования: в первую очередь выполняются процессы с большим приоритетом, а процессы с одинаковыми приоритетами выполняются в течении кванта времени циклически друг за другом. Если процесс, имеющий более высокий приоритет, поступает в очередь готовых к выполнению, планировщик вытесняет текущий процесс и предоставляет ресурс более приоритетному.

Приоритет представляет собой целое число из диапазона от 0 до 127. Чем меньше число, тем выше приоритет:
\begin{itemize}
    \item в диапазоне от 0 до 49 находятся приоритеты ядра;
    \item в диапазоне от 50 до 127 – приоритеты прикладных задач.
\end{itemize}

Приоритеты ядра являются фиксированными величинами.

Приоритеты прикладных задач могут изменяться во времени в зависимости от следующих двух факторов:
\begin{itemize}
    \item фактор любезности;
    \item последней измеренной величины использования процессора.
\end{itemize}

Фактор любезности – целое число в диапазоне от 0 до 39 со значением 20 по умолчанию. Чем меньше значение фактора любезности, тем выше приоритет процесса. Фоновым процессам автоматически задаются более высокие значения этого фактора. Фактор любезности процесса может быть изменен суперпользователем с помощью системного вызова nice.

Дескриптор процесса \texttt{proc} содержит следующие поля, относящиеся к приоритету:
\begin{itemize}
    \item \texttt{p\_pri} – текущий приоритет планирования;
    \item \texttt{p\_usrpri} – приоритет процесса в режиме задачи;
    \item \texttt{p\_cpu} – результат последнего измерения степени загруженности процессора процессом;
    \item \texttt{p\_nice} – фактор любезности.
\end{itemize}

У процесса, находящегося в режиме задачи, значения \texttt{p\_pri} и \texttt{p\_usrpri} равны. Значение текущего приоритета \texttt{p\_pri} может быть повышено планировщиком для выполнения процесса в режиме ядра (при этом \texttt{p\_usrpri} будет использоваться для хранения приоритета, который будет назначен при возврате в режим задачи)

Ядро системы связывает приоритет сна с событием или ожидаемым ресурсом, из-за которого процесс может блокироваться. Когда процесс просыпается после блокирования в системном вызове, ядро устанавливает в поле \texttt{p\_pri} приоритет сна – значение приоритета из диапазона от 0 до 49, зависящее от события или ресурса по которому произошла блокировка. Событие и связанное с ним значение приоритета сна в системе \texttt{4.3BSD} описывает таблица \ref{tab:bsd}.

\begin{table}[h]
    \caption{Таблица приоритетов в системе \texttt{4.3BSD}}
    \label{tab:bsd}
    \begin{center}
        \begin{tabular}{ |c|c|c| }
            \hline
            \textbf{Приоритет} & \textbf{Значение} & \textbf{Описание} \\
            \hline
            \texttt{PSWP} & 0 & Свопинг \\
            \hline
            \texttt{PSWP + 1} & 1 & Страничный демон \\
            \hline
            \texttt{PSWP + 1/2/4} & 1/2/4 & Другие действия по обработке памяти \\
            \hline
            \texttt{PINOD} & 10 & Ожидание освобождения inode \\
            \hline
            \texttt{PRIBIO} & 20 & Ожидание дискового ввода-вывода \\
            \hline
            \texttt{PRIBIO + 1} & 21 & Ожидание освобождения буфера \\
            \hline
            \texttt{PZERO} & 25 & Базовый приоритет \\
            \hline
            \texttt{TTIPRI} & 28 & Ожидание ввода с терминала \\
            \hline
            \texttt{TTOPRI} & 29 & Ожидание вывода с терминала \\
            \hline
        \end{tabular}
    \end{center}
\end{table}

При создании процесса поле \texttt{p\_cpu} инициализируется нулем. На каждом тике обработчик таймера увеличивает поле \texttt{p\_cpu} текущего процесса на единицу, до максимального значения, равного 127. Каждую секунду, обработчик прерывания инициализирует отложенный вызов процедуры \texttt{schedcpy()}, которая уменьшает значение \texttt{p\_cpu} каждого процесса исходя из фактора \textit{``полураспада''}.

В системе \texttt{4.3BSD} для расчёта фактора полураспада применяется формула \eqref{for:bsd}.

\begin{equation}
    \label{for:bsd}
    decay = \frac{2 \cdot load\_average}{2 \cdot load\_average + 1}
\end{equation}

где \texttt{load\_average} - это среднее количество процессов, находящихся в состоянии готовности к выполнению, за последнюю секунду.

Процедура \texttt{schedcpy()} пересчитывает приоритеты для режима задачи всех процессов по формуле \eqref{for:sc}.

\begin{equation}
    \label{for:sc}
    p\_usrpri = PUSER + \frac{p\_cpu}{2} + 2 \cdot p\_nice
\end{equation}

где \texttt{PUSER} - базовый приоритет в режиме задачи, равный 50.

В результате, если процесс в последний раз использвоал большое количство процессорного времени, его \texttt{p\_cpu} будет увеличен. Это приведёт к росту значения \texttt{p\_usrpri} и, следовательно, к понижению приоритета. Чем дольше процесс простаивает в очереди на исполнение, тем больше фактор полураспада уменьшает его \texttt{p\_cpu}, что приводит к повышению его приоритета. Такая схема предотвращает зависание низкоприоритетных процессов по вине операционной системы. Её применение предпочтительнее процессам, осуществляющим много операций ввода-вывода, в противоположность процессам, производящим много вычислений.



\section{Пересчет динамических приоритетов в операционных системах семейства Windows}

В Windows при создании процесса, ему назначается базовый приоритет. Относительно базового приоритета процесса потоку назначается относительный приоритет.

Планирование осуществляется на основании приоритетов потоков, готовых к выполнению. Поток с более низким приоритетом вытесняется планировщиком, когда поток с более высоким приоритетом становится готовым к выполнению. По истечению кванта времени текущего потока, ресурс передается первому --- самому приоритетному --- потоку в очереди готовых на выполнение.

Раз в секунду диспетчер настройки баланса сканирует очередь готовых потоков. Если обнаружены потоки, ожидающие выполнения более 4 секунд, диспетчер настройки баланса повышает их приоритет до 15. Как только квант истекает, приоритет потока снижается до базового приоритета. Если поток не был завершен за квант времени или был вытеснен потоком с более высоким приоритетом, то после снижения приоритета поток возвращается в очередь готовых потоков.

Чтобы минимизировать расход процессорного времени, диспетчер настройки баланса сканирует лишь 16 готовых потоков. Кроме того, диспетчер повышает приоритет не более чем у 10 потоков за один проход: обнаружив 10 потоков, приоритет которых следует повысить, он прекращает сканирование. При следующем проходе сканирование возобновляется с того места, где оно было прервано в прошлый раз. Наличие 10 потоков, приоритет которых следует повысить, говорит о необычно высокой загруженности системы.


В Windows используется 32 уровня приоритета: целое число от 0 до 31, где 31 --- наивысший приоритет, из них:
\begin{itemize}
    \item от 16 до 31 --- уровни реального времени;
    \item от 0 до 15 --- динамические уровни, уровень 0 зарезервирован для потока обнуления страниц.
\end{itemize}

Уровни приоритета потоков назначаются \texttt{Windows API} и ядром операционной системы.

\texttt{Windows API} сортирует процессы по классам приоритета, которые были назначены при их создании:
\begin{itemize}
    \item реального времени (real-time, 4);
    \item высокий (high, 3);
    \item выше обычного (above normal, 6);
    \item обычный (normal, 2);
    \item ниже обычного (below normal, 5);
    \item простой (idle, 1).
\end{itemize}

Затем назначается относительный приоритет потоков в рамках процессов:

\begin{itemize}
    \item критичный по времени (time critical, 15);
    \item наивысший (highest, 2);
    \item выше обычного (above normal, 1);
    \item обычный (normal, 0);
    \item ниже обычного (below normal, -1);
    \item низший (lowest, -2);
    \item простой (idle, -15).
\end{itemize}

Исходный базовый приоритет потока наследуется от базового приоритета процесса. Процесс по умолчанию наследует свой базовый приоритет у того процесса, который его создал.

Соответствие между приоритетами \texttt{Windows API} и ядра системы приведено в таблице \ref{tbl:priority}.
\clearpage

\begin{table}[h]
    \caption{Соответствие между приоритетами Windows API и ядра Windows}
    \begin{center}
        \begin{tabular}{|l|p{45pt}|p{45pt}|p{45pt}|p{45pt}|p{45pt}|p{45pt}|}
            \hline
            {} & \textbf{real-time} & \textbf{high} & \textbf{above normal} & \textbf{normal} & \textbf{below normal} & \textbf{idle}\\
            \hline
            \textbf{time critical} & 31 & 15 & 15 & 15 & 15 & 15 \\
            \hline
            \textbf{highest} & 26 & 15 & 12 & 10 & 8 & 6 \\
            \hline
            \textbf{above normal} & 25 & 14 & 11 & 9 & 7 & 5 \\
            \hline
            \textbf{normal} & 24 & 13 & 10 & 8 & 6 & 4 \\
            \hline
            \textbf{below normal} & 23 & 12 & 9 & 7 & 5 & 3 \\
            \hline
            \textbf{lowest} & 22 & 11 & 8 & 6 & 4 & 2 \\
            \hline
            \textbf{idle} & 16 & 1 & 1 & 1 & 1 & 1 \\
            \hline
        \end{tabular}
    \end{center}
    \label{tbl:priority}
\end{table}


Текущий приоритет потока в динамическом диапазоне --- от 1 до 15 --- может быть повышен планировщиком вследствие следующих причин:

\begin{itemize}
    \item повышение вследствие событие планировщика или диспетчера;
    \item повышение приоритета владельца блокировки;
    \item повышение приоритета после завершения ввода/вывода (см. таблицу \ref{tab:io});
    \item повышение приоритета вследствие ввода из пользовательского интерфейса;
    \item повышение приоритета вследствие длительного ожидания ресурса исполняющей системы;
    \item повышение вследствие ожидания объекта ядра;
    \item повышение приоритета в случае, когда готовый к выполнению поток не был запущен в течение длительного времени;
    \item повышение приоритета проигрывания мультимедиа службой планировщика \texttt{MMCSS}.
\end{itemize}


\begin{table}[h]
    \caption{Рекомендуемые значения повышения приоритета.}
    \begin{center}
        \begin{tabular}{|p{100mm}|l|}
            \hline
            \textbf{Устройство} & \textbf{Приращение} \\
            \hline
            Диск, CD-ROM, параллельный порт, видео & 1 \\
            \hline
            Сеть, почтовый ящик, именованный канал, последовательный порт & 2 \\
            \hline
            Клавиатура, мышь & 6 \\
            \hline
            Звуковая плата & 8 \\
            \hline
        \end{tabular}
    \end{center}
    \label{tab:io}
\end{table}

Текущий приоритет потока в динамическом диапазоне может быть понижен до базового приоритета путем вычитания всех повышений.

\subsection{MMCSS}

Мультимедийные потоки должны выполняться с минимальными задержками. Эта задача решена в Windows путем повышения приоритетов мультимедийных потоков драйвером \texttt{MultiMedia Class Scheduler Service} (\texttt{MMCSS}). Повышение приоритетов мультимедийных потоков происходит следующим образом: приложения, которые реализуют воспроизведение мультимедийного контента, указывают драйверу \texttt{MMCSS} задачу из следующего списка:

\begin{itemize}
    \item аудио;
    \item захват;
    \item распределение;
    \item игры;
    \item воспроизведение;
    \item задачи администратора многоэкранного режима.
\end{itemize}

Одно из наиболее важных свойств для планирования потоков называется категорией планирования --- является первичным фактором, определяющим приоритет потоков, зарегистрированных c \texttt{MMCSS}. В таблице \ref{tab:plan} показаны различные категории планирования.

\begin{table}[h]
    \caption{Категории планирования.}
    \begin{center}
        \begin{tabular}{|p{40mm}|p{30mm}|p{80mm}|}
            \hline
            \textbf{Категория} & \textbf{Приоритет} & \textbf{Описание} \\
            \hline
            High (Высокая) & 23-26 & Потоки профессионального аудио (Pro Audio), запущенные с приоритетом выше, чем у других потоков на системе, за исключением критических системных потоков \\
            \hline
            Medium (Средняя) & 16-22 & Потоки, являющиеся частью приложений первого плана, например Windows Media Player \\
            \hline
            Low (Низкая) & 8-15 & Все остальные потоки, не являющиеся частью предыдущих категорий \\
            \hline
            Exhausted (Исчерпавших потоков) & 1-7 & Потоки, исчерпавшие свою долю времени центрального процессора, выполнение которых продолжиться, только если не будут готовы к выполнению другие потоки с более высоким уровнем приоритета \\
            \hline
        \end{tabular}
    \end{center}
    \label{tab:plan}
\end{table}

Функции \texttt{MMCSS} временно повышают приоритет потоков, зарегистрированных с \texttt{MMCSS} до уровня, соответствующего их категориям планирования.  Затем, их приоритет снижается до уровня, соответствующего категории \texttt{Exhausted}, чтобы другие потоки также могли получить ресурс.

\chapter*{Вывод}
\addcontentsline{toc}{chapter}{Вывод}

Функции обработчика прерывания от системного таймера в защищенном режиме для семейства ОС \texttt{Windows} и для семейства OC \texttt{UNIX/Linux} очень похожи по своим действиям. Они выполняют схожие задачи:

\begin{itemize}
    \item инициализируют (но не выполняют) отложенные действия, относящиеся к работе планировщика, такие как пересчет приоритетов;
    \item выполняют декремент счетчиков времени: часов, таймеров, будильников реального времени, счетчиков времени отложенных действий.
    \item выполняют декремент кванта (текущего процесса в \texttt{Linux}, текущего потока в \texttt{Windows}).
\end{itemize}

Обе системы являются системами разделения времени с динамическими приоритетами и вытеснением, пересчёт динамических приоритетов в данных системах можно описать следующим образом:
\begin{itemize}
    \item В \texttt{UNIX/Linux} приоритет процесса характеризуется текущим приоритетом и приоритетом процесса в режиме задачи. Приоритет пользовательского процесса --- процесса в режиме задачи --- может быть динамически пересчитан в зависимости от фактора любезности и величины использования процессора, в то время как приоритеты ядра являются фиксированными величинами.
    \item При создании процесса в \texttt{Windows}, ему назначается приоритет, обычно называемый базовым. Приоритеты потоков определяются относительно приоритета процесса, в котором они создаются. Приоритет потока пользовательского процесса может быть пересчитан динамически.
\end{itemize}
